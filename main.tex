\documentclass[xcolor={x11names}]{beamer}
\usetheme{Madrid}

\usepackage{amssymb}
\usepackage{ulem}
\usepackage[utf8]{inputenc}
\usepackage{mathtools}
\usepackage{multicol}
%\usepackage[x11names]{xcolor}
\usefonttheme{professionalfonts}


% Subfigures
\usepackage{caption}
\usepackage{subcaption}


% License
\usepackage[
    type={CC},
    modifier={by-nc-sa},
    version={4.0},
    imagewidth=4pt
]{doclicense}




% Change base colour beamer@blendedblue (originally RGB: 0.2,0.2,0.7)
\colorlet{beamer@blendedblue}{DarkSeaGreen4}







%% MATH commands
\DeclareMathOperator{\Var}{Var}


%% THEOREMS
%\newtheorem{theorem}{Theorem}
\newtheorem{thm}{Teorema}[section] % the main one
% Definición
%\theoremstyle{definition}
\newtheorem{definicion}{Definición}[section]
\newtheorem{lema}{Lema}[section]



%% PFGplots %%
\usepackage{pgfplots}

%% Exponential distribution
\pgfmathdeclarefunction{exponential}{1}{%
  \pgfmathparse{(#1)*exp(-#1*x)}%
}
\pgfmathdeclarefunction{exponentialcdf}{1}{%
  \pgfmathparse{1-exp(-#1*x)}%
}

%% Poisson distribution
\pgfmathdeclarefunction{poiss}{1}{%
  \pgfmathparse{(#1^x)*exp(-#1)/(x!)}%
}

%% Normal distribution (#1=mu, #2=sigma)
% John D. Cook approx. https://tex.stackexchange.com/a/124629
\pgfmathdeclarefunction{normalcdf}{2}{%
  \pgfmathparse{1/(1 + exp(-0.07056*((x-#1)/#2)^3 - 1.5976*(x-#1)/#2))}%
}




\newcommand{\red}[1]{{\color{red}#1}}
\newcommand{\blue}[1]{{\color{blue}#1}}

%%%%%%%%%%
%% TIKZ %%
%%%%%%%%%%
\usepackage{tikz}
\usepackage{animate}
\usetikzlibrary{positioning}
\usetikzlibrary{shapes,arrows, positioning, calc}
\usetikzlibrary{overlay-beamer-styles}
\usetikzlibrary{chains,shapes.multipart}
\usetikzlibrary{scopes}
\usetikzlibrary{automata}
\usetikzlibrary{positioning}  %                 ...positioning nodes
\usetikzlibrary{arrows}       %                 ...customizing arrows
\usetikzlibrary{intersections}


%%%%%%%%%
%% PGF %%
%%%%%%%%%
\usepgfplotslibrary{fillbetween}


%%% Insert section name before the section %%%
\AtBeginSection[]{
  \begin{frame}
  \vfill
  \centering
  \begin{beamercolorbox}[sep=8pt,center,shadow=true,rounded=true]{title}
    \usebeamerfont{title}\insertsectionhead\par%
  \end{beamercolorbox}
  \vfill
  \end{frame}
}



\title[Tema 4]{Tema 4: Cálculo de Redes}
%% \subtitle{Redes y Servicios de Telecomunicaciones (RSTC)\\
%% Grado en Ingeniería de Tecnologías y Servicios de Telecomunicación}
%\author{M. Saiful Bari\inst{1} \and Mr X\inst{2}}

\titlegraphic{%
\includegraphics[height=2cm]{figs/RSTC-grande.png}\\%
\doclicenseIcon {\tiny \hspace{1em}\doclicenseText}\\%
\href{https://github.com/MartinPJorge/RSTC-netcal-slides}{\includegraphics[height=1cm]{figs/github-logo.png}}%
}

\author{\textcolor{white}{RSTC curso 2024-2025}}
%\author{Jorge Martín Pérez\inst{1}}
%\institute{
%    \inst{1}
%    Departamento de Ingeniería Telemática, Universidad Politécnica de Madrid
%}

\date{\today}







%%%%%%%%%%%%%%%%%%%%
%%% SLIDES START %%%
%%%%%%%%%%%%%%%%%%%%
\begin{document}


%%% TITLE %%%
\frame{\titlepage }


\begin{frame}[allowframebreaks]{Contenido}
    \tableofcontents
\end{frame}




\section{Introducción}
\begin{frame}{\secname}
    El cálculo de redes modela flujos
    \begin{itemize}
        \item electricidad
        \item fluidos
        \item \textbf{tráfico en internet}
    \end{itemize}

    \vfill
    Nos sirve para modelar:
    \begin{itemize}
        \item conformado de tráfico
        \item políticas de tráfico
        \item averiguar métricas de
            lantencia y tamaño en cola
    \end{itemize}
\end{frame}



\section{Álgebra min-plus}
\begin{frame}{\secname}
    \begin{definicion}[Álgebra min-plus]
        Es un diodo\footnote{Un tipo de
        estructura algebraica.}
        definido en
        $(\mathbb{R}\cup \{+\infty\},
        \land, +)$, donde:
        \begin{itemize}
            \item $\land$ es el operador
                $\min$
            \item $+$ es la suma
        \end{itemize}
    \end{definicion}

    \vfill

    \emph{Ejemplo}:

    la operación
    \begin{equation*}
        (1+2)\cdot 3 = 9
    \end{equation*}
    se ``traduce'' en:
    \begin{equation*}
        (1\land2)+ 3 = 4
    \end{equation*}
\end{frame}



\begin{frame}{\secname}
    \begin{definicion}[Familia de funciónes crecientes]
        Sea $\mathcal{F}$ el conjunto 
        de funciones crecientes, decimos
        que $f\in\mathcal{F}$ es una función
        creciente definida en
        $f:\mathbb{R}^+\mapsto\mathbb{R}^+$
        si y sólo si cumple:
        \begin{equation}
            f(s)\geq f(t), \forall s\geq t
        \end{equation}
    \end{definicion}

    \vfill

    \emph{Ejemplo}: la función ``rate-latency''
    $\beta_{R,T}(t)=R[t-T]^+$



    \begin{figure}[h]
        \centering
        \begin{tikzpicture}[
  declare function={
    ratelatency(\R,\T,\x)= (\x<=\T) * 0   +
    (\x>\T) * (\R*(\x-\T));
    }
]
\begin{axis}[
    x=1cm,
    y=1cm,
    domain=0:5,
    ymin=0,
    xmin=0,
    axis x line=middle,
    axis y line=middle,
     xlabel style={
            anchor=west,
            at={(ticklabel* cs:1.0)},
            xshift=5pt
        },
        xlabel=$t$,
    ]
    \addplot[DodgerBlue3,ultra thick,smooth]
        {ratelatency(0.5,1,x)};
\end{axis}
\end{tikzpicture}

    \end{figure}

\end{frame}





\end{document}
