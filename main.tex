\documentclass[xcolor={x11names}]{beamer}
\usetheme{Madrid}

\usepackage{amssymb}
\usepackage{mdframed}
\usepackage{ulem}
\usepackage[utf8]{inputenc}
\usepackage{mathtools}
\usepackage{multicol}
%\usepackage[x11names]{xcolor}
\usefonttheme{professionalfonts}


% Subfigures
\usepackage{caption}
\usepackage{subcaption}


% License
\usepackage[
    type={CC},
    modifier={by-nc-sa},
    version={4.0},
    imagewidth=4pt
]{doclicense}




% Change base colour beamer@blendedblue (originally RGB: 0.2,0.2,0.7)
\colorlet{beamer@blendedblue}{DarkSeaGreen4}







%% MATH commands
\DeclareMathOperator{\Var}{Var}


%% THEOREMS
%\newtheorem{theorem}{Theorem}
\newtheorem{thm}{Teorema}[section] % the main one
% Definición
%\theoremstyle{definition}
\newtheorem{definicion}{Definición}[section]
\newtheorem{lema}{Lema}[section]



%% PFGplots %%
\usepackage{pgfplots}

%% Exponential distribution
\pgfmathdeclarefunction{exponential}{1}{%
  \pgfmathparse{(#1)*exp(-#1*x)}%
}
\pgfmathdeclarefunction{exponentialcdf}{1}{%
  \pgfmathparse{1-exp(-#1*x)}%
}

%% Poisson distribution
\pgfmathdeclarefunction{poiss}{1}{%
  \pgfmathparse{(#1^x)*exp(-#1)/(x!)}%
}

%% Normal distribution (#1=mu, #2=sigma)
% John D. Cook approx. https://tex.stackexchange.com/a/124629
\pgfmathdeclarefunction{normalcdf}{2}{%
  \pgfmathparse{1/(1 + exp(-0.07056*((x-#1)/#2)^3 - 1.5976*(x-#1)/#2))}%
}




\newcommand{\red}[1]{{\color{red}#1}}
\newcommand{\blue}[1]{{\color{blue}#1}}

%%%%%%%%%%
%% TIKZ %%
%%%%%%%%%%
\usepackage{tikz}
\usepackage{animate}
\usetikzlibrary{positioning}
\usetikzlibrary{shapes,arrows, positioning, calc}
\usetikzlibrary{overlay-beamer-styles}
\usetikzlibrary{chains,shapes.multipart}
\usetikzlibrary{scopes}
\usetikzlibrary{automata}
\usetikzlibrary{positioning}  %                 ...positioning nodes
\usetikzlibrary{arrows}       %                 ...customizing arrows
\usetikzlibrary{intersections}


%%%%%%%%%
%% PGF %%
%%%%%%%%%
\usepgfplotslibrary{fillbetween}


%%% Insert section name before the section %%%
\AtBeginSection[]{
  \begin{frame}
  \vfill
  \centering
  \begin{beamercolorbox}[sep=8pt,center,shadow=true,rounded=true]{title}
    \usebeamerfont{title}\insertsectionhead\par%
  \end{beamercolorbox}
  \vfill
  \end{frame}
}



\title[Tema 4]{Tema 4: Cálculo de Redes}
%% \subtitle{Redes y Servicios de Telecomunicaciones (RSTC)\\
%% Grado en Ingeniería de Tecnologías y Servicios de Telecomunicación}
%\author{M. Saiful Bari\inst{1} \and Mr X\inst{2}}

\titlegraphic{%
\includegraphics[height=2cm]{figs/RSTC-grande.png}\\%
\doclicenseIcon {\tiny \hspace{1em}\doclicenseText}\\%
\href{https://github.com/MartinPJorge/RSTC-netcal-slides}{\includegraphics[height=1cm]{figs/github-logo.png}}%
}

\author{\textcolor{white}{RSTC curso 2024-2025}}
%\author{Jorge Martín Pérez\inst{1}}
%\institute{
%    \inst{1}
%    Departamento de Ingeniería Telemática, Universidad Politécnica de Madrid
%}

\date{\today}







%%%%%%%%%%%%%%%%%%%%
%%% SLIDES START %%%
%%%%%%%%%%%%%%%%%%%%
\begin{document}


%%% TITLE %%%
\frame{\titlepage }


\begin{frame}[allowframebreaks]{Contenido}
    \tableofcontents
\end{frame}




\section{Introducción}
\begin{frame}{\secname}
    El cálculo de redes modela flujos
    \begin{itemize}
        \item electricidad
        \item fluidos
        \item \textbf{tráfico en internet}
    \end{itemize}

    \vfill
    Nos sirve para modelar:
    \begin{itemize}
        \item conformado de tráfico
        \item políticas de tráfico
        \item averiguar métricas de
            lantencia y tamaño en cola
    \end{itemize}
\end{frame}



\section{Álgebra min-plus}
\begin{frame}{\secname}
    \begin{definicion}[Álgebra min-plus]
        Es un diodo\footnote{Un tipo de
        estructura algebraica.}
        definido en
        $(\mathbb{R}\cup \{+\infty\},
        \land, +)$, donde:
        \begin{itemize}
            \item $\land$ es el operador
                $\min$
            \item $+$ es la suma
        \end{itemize}
    \end{definicion}

    \vfill

    \emph{Ejemplo}:

    la operación
    \begin{equation*}
        (1+2)\cdot 3 = 9
    \end{equation*}
    se ``traduce'' en:
    \begin{equation*}
        (1\land2)+ 3 = 4
    \end{equation*}
\end{frame}



\begin{frame}{\secname}
    \begin{definicion}[Familia de funciónes crecientes]
        Sea $\mathcal{F}$ el conjunto 
        de funciones crecientes, decimos
        que $f\in\mathcal{F}$ es una función
        creciente definida en
        $f:\mathbb{R}^+\mapsto\mathbb{R}^+$
        si y sólo si cumple:
        \begin{equation}
            f(s)\geq f(t), \forall s\geq t
        \end{equation}
        y además
        $f(t)=0,\forall t<0$.
    \end{definicion}

    \vfill

    \emph{Ejemplo}: la función ``rate-latency''
    $\beta_{R,T}(t)=R[t-T]^+$



    \begin{figure}[h]
        \centering
        \begin{tikzpicture}[
  declare function={
    ratelatency(\R,\T,\x)= (\x<=\T) * 0   +
    (\x>\T) * (\R*(\x-\T));
    }
]
\begin{axis}[
    x=1cm,
    y=1cm,
    domain=0:5,
    ymin=0,
    xmin=0,
    axis x line=middle,
    axis y line=middle,
     xlabel style={
            anchor=west,
            at={(ticklabel* cs:1.0)},
            xshift=5pt
        },
        xlabel=$t$,
    ]
    \addplot[DodgerBlue3,ultra thick,smooth]
        {ratelatency(0.5,1,x)};
\end{axis}
\end{tikzpicture}

    \end{figure}

\end{frame}


\subsection{Convolución}
\begin{frame}{\secname: \subsecname}
    \begin{definicion}[Convolución min-plus]
        La convolución
        min-plus
        $\otimes$ de dos
        funciones crecientes
        $f,g\in\mathcal{F}$
        se define como
        \begin{equation}
            (f\otimes g)(t)
            =\inf_{0\leq s\leq t}
            \{f(t-s)+g(s)\}
        \end{equation}
    \end{definicion}

    \vfill

    \emph{Nota}: equivalente
    a la convolución
    ``clásica'':
    \begin{equation*}
        (f\ast g)(t)
        =\int_0^t
        f(t-s)g(s)\ ds
    \end{equation*}
\end{frame}






\begin{frame}{\secname: \subsecname}
    Si $f(0)=g(0)=0$ se puede
    calcular
    comenzando a dibujar
    cada función sobre
    todo punto de la
    otra y tomando
    el mínimo.
    \vfill
    \begin{figure}[h]
        \centering
        \begin{tikzpicture}[
  declare function={
    ratelatency(\R,\T,\x)= (\x<=\T) * 0   +
    (\x>\T) * (\R*(\x-\T));
    }
]
\begin{axis}[
    x=1cm,
    y=1cm,
    domain=0:5,
    ymin=0,
    xmin=0,
    axis x line=middle,
    axis y line=middle,
     xlabel style={
            anchor=west,
            at={(ticklabel* cs:1.0)},
            xshift=5pt
        },
    xlabel=$t$,
    legend pos=outer north east
    ]
    \addplot[DodgerBlue3,ultra thick,smooth]
        {ratelatency(0.75,2,x)};
    \addplot[Firebrick3,ultra thick,smooth]
        {0.4*x};

    \addplot[Gold3,ultra thick,smooth]
        {ratelatency(0.4,2,x)};

    \foreach \t in {.5,1.5,2.5,3.5,4.5}  {
        
        \addplot[DodgerBlue2,smooth,domain=\t:5]
        {ratelatency(0.75,2,x-\t)+0.4*\t};
    }


    \legend{$f(t)$,
    $g(t)$,
    $(f\otimes g)(t)$};

\end{axis}
\end{tikzpicture}

    \end{figure}


    \vfill

    En este ejemplo:
    $f(t)=rt,
    g(t)=\beta_{R,T}(t)$
    con $R>r>0$.
\end{frame}



\begin{frame}{\secname: \subsecname}
    Calculemos la
    convolución min-plus
    de manera analítica:
    \begin{equation*}
        (f\otimes g)(t)
        =\inf_{0\leq s \leq t}
        \{f(t-s)+g(s)\}
        =\inf_{0\leq s \leq t}
        \{r(t-s)+R[s-T]^+\}
    \end{equation*}

    \pause

    \emph{Con $T\geq t$}:
    \begin{equation*}
        (f\otimes g)(t)
        =\inf_{0\leq s \leq t}
        \{r(t-s)+0\}=0
    \end{equation*}

    \pause

    \emph{Con $T< t$}
    dividimos en dos casos y
    tomamos el menor
        \pause
    \begin{enumerate}
        \item $0\leq s\leq T$:
        \begin{equation*}
            (f\otimes g)(t)
        =\inf_{0\leq s \leq T < t}
        \{r(t-s)+0\}=r(t-T)
        \end{equation*}
        \pause
    \item $T<s$:
        \begin{equation*}
            (f\otimes g)(t)
        =\inf_{T< s \leq t}
        \{r(t-s)+R(s-T)\}
        =r(t-T)
        \end{equation*}
    \end{enumerate}

    \pause
    El resultado es:
    \begin{mdframed}[rightmargin=20em]
        $(f\otimes g)(t)
        =r[t-T]^+$.
    \end{mdframed}

\end{frame}



\begin{frame}{\secname: EJercicio}
Calcule la convolución min-plus
de las función rate-burst
$\gamma_{r,b}(t)=rt+b$
y la función rate-latency
$\beta_{R,T}(t)=R[t-T]^+$.
Obtenga la solución de manera analítica.


Tenemos:
\begin{equation*}
(\gamma_{r,b}\otimes \beta_{R,T})(t)
=\inf_{0\leq s \leq t}\{r(t-s)+b+R[s-T]^+\}
\end{equation*}
Con $t\leq T$ tenemos
\begin{equation*}
(\gamma_{r,b}\otimes \beta_{R,T})(t)
=\inf_{0\leq s \leq t}\{r(t-s)+b+0\}=b
\end{equation*}
y con $t>T$ tenemos dos casos.
El primero es $s<T<t$:
\begin{equation*}
(\gamma_{r,b}\otimes \beta_{R,T})(t)
=\inf_{0\leq s \leq T}\{r(t-s)+b+0\}=r(t-T)+b
\end{equation*}
y el segundo es $T<s\leq t$
\begin{equation*}
(\gamma_{r,b}\otimes \beta_{R,T})(t)
=\inf_{T< s \leq t}\{r(t-s)+b+R(s-T)\}=r(t-T)+b
\end{equation*}

Por tanto la solución es
\begin{equation*}
(\gamma_{r,b}\otimes \beta_{R,T})(t)
=r[t-T]^++b
\end{equation*}

\end{frame}




\begin{frame}{\secname: \subsecname}
    La convolución min-plus
    está dotada de las
    siguientes
    propiedades en $\mathcal{F}$:
    \begin{itemize}
        \item \textbf{Cierre}:
            $\forall f,g\in\mathcal{F},\quad f\otimes g\in\mathcal{F}$
        \pause
        \item \textbf{Asociativa}:
            $\forall f,g,h\in\mathcal{F},
            \quad 
            (f\otimes g)
            \otimes h
            =
            f\otimes
            (g\otimes h)$
        \pause
        \item \textbf{Conmutativa}:
            $\forall
            f,g\in\mathcal{F},
            \quad
            f\otimes g=
            g\otimes f$
        \pause
    \item \textbf{Elemento neutro\footnote{$\delta_0(t)=+\infty$ con $t>0$ y 0 con $t\leq0$}}:
            $\exists\delta_0\in\mathcal{F}: \forall f\in\mathcal{F},
            f\otimes\delta_0=f$
        \pause
    \item \textbf{Distrib. con $\land$}:
        $f,g,h\in\mathcal{F},
        \quad
        (f\land g)\otimes h
        = 
        (f\otimes h)
        \land
        (g\otimes h)$
    \pause
    \item \textbf{Suma constante}:
        $\forall f,g\in\mathcal{F},K\in\mathbb{R}^+,
        \quad
        (f+K)\otimes g
        =(f\otimes g) + K$

    \pause
    \item \textbf{Isotonicidad}:
        $\forall f,g,f',g'\in\mathcal{F},\quad
        f\leq f',g\leq g'
        \implies
        f\otimes f'
        \leq
        g\otimes g'$
    \end{itemize}
\end{frame}


\subsection{Deconvolución}
\begin{frame}{\secname: \subsecname}
    \begin{definicion}[Deconvolución min-plus]
        La deconvolución
        min-plus
        $\oslash$ de dos
        funciones crecientes
        $f,g\in\mathcal{F}$
        se define como
        \begin{equation}
            (f\oslash g)(t)
            =\sup_{u\geq0}
            \{f(t+u)-g(u)\}
        \end{equation}
    \end{definicion}

    \vfill

    \emph{Truco}:
    es como la convolución
    min-plus pero sustituyendo
    los $+$ por $-$.
\end{frame}




\begin{frame}{\secname: \subsecname}
    La deconvolución de
    $f(t)=\gamma_{r,b}(t)
    =rt+b$
    y $g(t)=\beta_{R,T}(t)$
    es\footnote{con $R>r$}
    \vfill
    \begin{figure}[h]
        \centering
        \input{figs/min-plus-deconv}
    \end{figure}


    \vfill

    \emph{Nota}: nótese
    que $(f\oslash g)(t)
    \notin \mathcal{F}$
    porque no es cero para
    $t<0$.
\end{frame}



\begin{frame}{\secname: \subsecname}
    Calculemos analíticamente
    la deconvolución:
    \begin{equation*}
        (\gamma_{r,b}
        \oslash
        \beta_{R,T})(t)
        =\sup_{u \geq0}
        \{
        \gamma_{r,b}(t+u)
        - \beta_{R,T}(u)
        \}
    \end{equation*}
    recordando que
    $\beta_{R,T}(u)=0,
    t\leq T$ dividimos en dos
    casos
    \pause
    \begin{equation*}
        (\gamma_{r,b}
        \oslash
        \beta_{R,T})(t)
        =\sup_{T>u \geq0}
        \{
        \gamma_{r,b}(t+u)
        \}
        \lor
        \sup_{u>T}
        \{
        \gamma_{r,b}(t+u)
        -
        \beta_{R,T}(u)
        \}
    \end{equation*}
    \pause
    en el primer caso
    $u$ llega a $T$
    y
    $\gamma_{r,b}(t+u)=0,
    t\leq-T$.
    Por tanto tenemos dos casos
    \begin{enumerate}
        \item $t\leq-T$:
    \begin{multline*}
        % (\gamma_{r,b}
        % \oslash
        % \beta_{R,T})(t)
        % =
        0
        \lor
        \sup_{-t\geq u>T}
        \{
        \gamma_{r,b}(t+u)
        -
        \beta_{R,T}(u)
        \}
        \lor
        \sup_{u>-t}
        \{
        \gamma_{r,b}(t+u)
        -
        \beta_{R,T}(u)
        \}\\
        = 0
        \lor
        \sup_{-t\geq u>T}
        \{0-Ru+RT\}
        \lor
        \sup_{u>-t}\{ r(t+u)+b
        -R(u-T)\}\\
        =0\lor0\lor
        \{b+R(t+T)\}
        =[b+R(t+T)]^+
    \end{multline*}
    \end{enumerate}

\end{frame}



\begin{frame}{\secname: \subsecname}
    \begin{enumerate}
        \setcounter{enumi}{1}
        \item $t>-T$:
        \begin{multline*}
        (\gamma_{r,b}
        \oslash
        \beta_{R,T})(t)
        =\sup_{T>u \geq0}
        \{
        \gamma_{r,b}(t+u)
        \}
        \lor
        \sup_{u>T}
        \{
        \gamma_{r,b}(t+u)
        -
        \beta_{R,T}(u)
        \}\\
        =\{ r(t+T)+b \}
        \lor
        \sup_{u>T}\{
            r(t+u)+b
            -R(u-T)
        \}\\
        =\{ r(t+T)+b \}
        \lor
        \sup_{u>T}\{
            (r-R)u+b
            +rt+RT
        \}\\
        =\{ r(t+T)+b \}
        \lor
        \{
            r(t+T)+b
        \}
        =r(t+T)+b
        \end{multline*}
    \end{enumerate}

    \pause
    Como resultado se obtiene
    \begin{equation*}
        (\gamma_{r,b}
        \oslash
        \beta_{R,T})(t)
        =
        \begin{cases}
            [b+R(t+T)]^+,\quad t\leq -T\\
            r(t+T)+b,\quad t>-T
        \end{cases}
    \end{equation*}
\end{frame}



\end{document}
